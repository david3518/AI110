\documentclass[journal,12pt,twocolumn]{IEEEtran}

\usepackage{enumitem}
\usepackage{amsmath}
\usepackage{amssymb}
\usepackage{graphicx}


\title{Assignment 4 \\ \Large AI1110: Probability and Random Variables \\ \large Indian Institute of Technology Hyderabad}
\author{MALOTH DAVID \\ \normalsize CS21BTECH11035\\ \vspace*{20pt} \normalsize  june 12 \\ \vspace*{20pt} \Large paupolis ex 9.28}


\begin{document}
	% The title
	\maketitle
	
	% The question
	\textbf{(paupolis Exercise 9.28)}  \textbf{(Papoulis chapter 9 ,ex 9.28 )} show that $R_X_X(t_1,t_2)$ =$q(t_1)\delta(t_1,t_2)$  $E\{y^2(t)\} = I(t)$  a) $y(t) = \int_0^I h(t,\alpha)X(\alpha)d\alpha $ then $I(t)=\int_0^I h^2(t,\alpha)q(\alpha)d\alpha$
	b)$y'(t)+c(t)y(t)=x(t) then I'(t)+2c(t)I(t)=q(t)$

% Blocks frame
\section{Solution}
\begin{frame}{Solution}

\begin{itemize}

\item a) $I(t) = E{\int_0^t\int_0^t h(t,\alpha)x(\alpha)h(t,\beta)x(\beta)d\alpha d\beta }$

\item $I(t) = \int_0^t\int_0^t h(t,\alpha)h(t,\alpha) q(\alpha)\delta(\alpha -\beta)d\alpha d\beta =\int_0^t h^2(t,\alpha)q(\alpha)d\alpha$

\item b) if $ y'(t)+c(t)y(t)=x(t)$ then y(t) is the output of a linear time varying system as in a) with impulse response h(t,$\alpha$) such that 

\item $\dfrac{dh(t,\alpha)}{dt} + c(t)h(t,\alpha) = \delta(t-\alpha)$ ,  $h(\alpha^-,\alpha)=0$

\item or equivalently
$\dfrac{dh(t,\alpha)}{dt} + c(t)h(t,\alpha) = 0$ ,$ t \textgreater 0$ ,$h(\alpha^+,\alpha)=1$

\item this yields
$h(t,\alpha) = e^-{\int_a^t c(\tau)d\tau}$

\item hence,if
$I(t) = \int_0^t h^2(t,\alpha)q(\alpha)d\alpha$ then $ I'(t)+2c(t)I(t)=q(t)$ 
\item because the impulse response of the of this equation equals
$e^-{2\int_a^tc(\tau)d\tau} = h^2(t,\alpha)$
\item hence proved
\end{itemize}
    
    
\end{frame} 



\end{document}
© 2022 GitHub, Inc.
Terms
Privacy
Security
Status
Docs
Co